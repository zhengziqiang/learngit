%%%%%%%%%%%%%%%%%%%%%%%%%%%%%%%%%%%%%%%%%%%%%%%%%%%%%%%%%%%%%%%%%%%%%%
% Overleaf (WriteLaTeX) Example: Molecular Chemistry Presentation
%
% Source: http://www.overleaf.com
%
% In these slides we show how Overleaf can be used with standard 
% chemistry packages to easily create professional presentations.
% 
% Feel free to distribute this example, but please keep the referral
% to overleaf.com
% 
%%%%%%%%%%%%%%%%%%%%%%%%%%%%%%%%%%%%%%%%%%%%%%%%%%%%%%%%%%%%%%%%%%%%%%
% How to use Overleaf: 
%
% You edit the source code here on the left, and the preview on the
% right shows you the result within a few seconds.
%
% Bookmark this page and share the URL with your co-authors. They can
% edit at the same time!
%
% You can upload figures, bibliographies, custom classes and
% styles using the files menu.
%
% If you're new to LaTeX, the wikibook is a great place to start:
% http://en.wikibooks.org/wiki/LaTeX
%
%%%%%%%%%%%%%%%%%%%%%%%%%%%%%%%%%%%%%%%%%%%%%%%%%%%%%%%%%%%%%%%%%%%%%%

\documentclass{beamer}

% For more themes, color themes and font themes, see:
% http://deic.uab.es/~iblanes/beamer_gallery/index_by_theme.html
%
\mode<presentation>
{
  \usetheme{Madrid}       % or try default, Darmstadt, Warsaw, ...
  \usecolortheme{default} % or try albatross, beaver, crane, ...
  \usefonttheme{serif}    % or try default, structurebold, ...
  \setbeamertemplate{navigation symbols}{}
  \setbeamertemplate{caption}[numbered]
} 

\usepackage[english]{babel}
\usepackage[utf8x]{inputenc}
\usepackage{chemfig}
\usepackage[version=3]{mhchem}

% On Overleaf, these lines give you sharper preview images.
% You might want to `comment them out before you export, though.
\usepackage{pgfpages}
\pgfpagesuselayout{resize to}[%
  physical paper width=8in, physical paper height=6in]

% Here's where the presentation starts, with the info for the title slide
\title[Vision@ouc]{Mid-term oral examination}
\author{Ziqiang Zheng}
\institute{OUC}
\date{May 25,2017}

\begin{document}

\begin{frame}
  \titlepage
\end{frame}

% These three lines create an automatically generated table of contents.
\begin{frame}{Outline}
  \tableofcontents
\end{frame}

\section{Introduction}

\begin{frame}{Introduction}

\begin{itemize}
  \item A sub topic of computer vision and fishery industry
  \item Main difficulties
  \item You can also find more quick tips and tricks on the help pages at \url{www.overleaf.com/help}
\end{itemize}

% Example from Chemfig documentation - Fischer indole synthesis:
% www.tex.ac.uk/ctan/macros/generic/chemfig/chemfig_doc_en.pdf
\begin{center}\small\setatomsep{1.5em}

\end{center}

\end{frame}

\subsection{The Main difficulties}
\begin{frame}{The Main difficulties}

We focus on two \LaTeX{} chemistry packages:
\begin{block}{The \texttt{chemfig} package}
This package provides the command which draws molecules. Created by Christian Tellechea, a detailed user guide can be found here:\\[0.4cm]
\small{\url{www.tex.ac.uk/ctan/macros/generic/chemfig/chemfig_doc_en.pdf}}
\end{block}
\begin{block}{The \texttt{mhchem} package}
The \texttt{mhchem} package provides simple commands for typesetting chemical molecular formulae and equations. Created by Martin Hensel, a detailed user guide can be found here:\\[0.4cm]
\small{\url{http://mirror.ox.ac.uk/sites/ctan.org/macros/latex/contrib/mhchem/mhchem.pdf}}
\end{block}
% The LaTeX wikibook is also a good source of info, e.g.
% http://en.wikibooks.org/wiki/LaTeX/Chemical_Graphics

\end{frame}

\section{Using chemistry packages with \LaTeX{}}

\subsection{Chemical equations with \texttt{mhchem}}

\begin{frame}[fragile]
\frametitle{Chemical equations with \texttt{mhchem}}

\begin{itemize}
\item The \texttt{mhchem} package lets you write chemical equations in \LaTeX{} with the minimum of effort. 
\item The example below shows how the standard representation of a reaction (on the left) is created from the simple code on the right:
\end{itemize}

\begin{center}
\ce{CO2 + C -> 2CO} is created with \verb|\ce{CO2 + C -> 2CO}|
\end{center}

\begin{itemize}
\item More complicated reactions are still easy to write:
\end{itemize}

\begin{center}
\ce{SO4^2- + Ba^2+ -> BaSO4 v}\\[0.1cm]
is created with\\[0.1cm]
\verb|\ce{SO4^2- + Ba^2+ -> BaSO4 v}|
\end{center}

\end{frame}

\subsection{Getting started with some \texttt{chemfig} coffee}

\begin{frame}[fragile]
\frametitle{Getting started with some \texttt{chemfig} coffee}

It's easy to use the \texttt{chemfig} package for drawing complex molecules:

\vskip 0.5cm

\begin{center}\small\setatomsep{2.0em}
\schemestart  
\chemfig{*6((=O)-N(-CH_3)-*5(-N=-N(-CH_3)-=)--(=O)-N(-H_3C)-)}
\schemestop
\end{center}

This is the caffeine molecule, represented clearly and neatly, and built from a single line of text: \small{\verb|\chemfig{*6((=O)-N(-CH_3)-*5(-N=-N(-CH_3)-=)--(=O)-N(-H_3C)-)}|}\\[0.3cm]

If that looks quite daunting, we can learn from simpler molecules\dots{}how about a single water molecule?

\end{frame}

\subsection{Experiments with water and rings}

\begin{frame}[fragile]
\frametitle{Experiments with water and rings}

To see how the \texttt{chemfig} package creates the drawings from your code, let us look at the simple water molecule:

\vskip 0.3cm
\begin{center} 
\chemfig{H_2O} is created with \verb|\chemfig{H_2O}|
\end{center}

The simple \LaTeX{} code on the right is automatically converted into the molecular formula for water on the left. 
\vskip 0.3cm
Rings are similarly easy to code - consider the examples below:

\vskip 0.3cm

\chemfig[][scale=0.5]{A*5(-B-C-D-E-)} = \verb|\chemfig{A*5(-B-C-D-E-)}|

\vskip 0.3cm

\chemfig[][scale=0.5]{*6(=-=-=-)} = \verb|\chemfig{*6(=-=-=-)}|


\end{frame}

\section{Where to go next\dots{}}

\begin{frame}{Where to go next\dots{}}

\begin{itemize}
\item This short example was designed to introduce you to using Overleaf for scientific presentations.
\item This is made possible by the many great packages that have been developed for \LaTeX{}, including the two we focused on here (plus the \texttt{Beamer} package used for the overall presentation style). 
\item For more help on using \LaTeX{}, see the links on the Overleaf help page: \url{www.overleaf.com/help} or check out our free introductory course: \url{www.overleaf.com/blog/7}.
\end{itemize}

\begin{center}
Follow @overleaf on Twitter for all the latest news and updates.\\[0.3cm]
Happy \LaTeX ing!
\end{center}

\end{frame}

\end{document}\grid
